\documentclass[a4paper,12pt]{extarticle}
\def\source{/home/osabio/tex/templates}
\input{\source/head.tex}
\def\labauthor{Сарафанов Ф.Г.}
\def\labauthors{Сарафанов Ф.Г.}
\def\labnumber{218}
\def\labtheme{Измерение емкости конденсатора}
\input{\source/math_left-of-equation.tex}
%%%%%%%%%%%%%%%%%%%%%%%%%%%%%%%%%%%%%%%%%%%%%%%%%
\geometry		
	{
		left			=	2cm,
		right 			=	2cm,
		top 			=	3cm,
		bottom 			=	3cm,
		bindingoffset	=	0cm
	}
\input{\source/lab_colontitle.tex}
\usepackage{epigraph}
% \setlength{\epigraphrule}{0pt}
\begin{document}

\input{\source/lab_title-page.tex} %Титульная страница

\tableofcontents
\newpage

\section*{Введение} % (fold)
\addcontentsline{toc}{section}{Введение}
\label{sec:input}
\epigraph{\textit{Существует лишь то, что можно измерить.}}
{Цитата приписывается Максу Планку}




 Для измерения сопротивлений, емкостей и индуктивностей  часто  применяют
компенсационный метод, заключающийся в компенсации измеряемой величины некой эталонной величиной. 

В схеме типа <<мост>> элементы цепи соединяют  <<четырехугольником>>, в одну диагональ  которого  включают  источник  напряжения,  а  в  другую  --
измерительный  прибор.  При  определенном  соотношении   между   параметрами
элементов измерительный прибор показывает отсутствие напряжения в  диагонали
(баланс моста).


\newpage
\section{Вывод формул}
\subsection{Напряжение на диагонали моста}

Применяя к контуру DATD второе правило Кирхгофа, получаем
\begin{equation}
i_1 R_1 + \frac{q_1}{C_1}=\varepsilon
\end{equation}
где$ i_1$ -- ток, текущий через сопротивление $R_1$, а $q_1$ -- заряд конденсатора $C_1$. 
Поскольку ток через измерительный прибор пренебрежимо мал ($R_G$ велико), то $i_1=\frac{\mathrm{d}q_1 }{\mathrm{d} t}$ и уравнение (1) принимает вид:
\begin{equation}
i_1=\frac{\mathrm{d}q_1 }{\mathrm{d} t} + \frac{q_1}{R_1 C_1}=\frac{\varepsilon}{R_1}
\end{equation}
Разделяя переменные и интегрируя:
\begin{equation}
 \int \limits^{q_1}_0 \frac{\mathrm{d} q_1}{q_1-\epsilon C_1}=-\int \limits^t_0 \frac{\mathrm{d} t}{R_1 C_1}
\end{equation}

\begin{equation}
	q(t)_1 = C_1 \varepsilon \cdot 
	% \left( 1- e ^{-\frac{t}{R_1 C_1}} \right)
	\left( 
		1-
		\exp\left[		
			-\frac{t}{R_1 C_1}
		\right]
	\right)
\end{equation}
Отсюда следует, что
\begin{equation}
	U_1(t) = \varepsilon \cdot 
	% \left( 1- e ^{-\frac{t}{R_1 C_1}}\right )
	\left( 
		1-
		\exp\left[		
			-\frac{t}{R_1 C_1}
		\right]
	\right)
\end{equation}
Аналогично рассматривая контур DBTD:
\begin{equation}
i_2 R_2 + \frac{q_2}{C_x}=\varepsilon
\end{equation}
\begin{equation}
i_2=\frac{\mathrm{d}q_2 }{\mathrm{d} t} + \frac{q_2}{R_2 C_x}=\frac{\varepsilon}{R_2}
\end{equation}
\begin{equation}
 \int \limits^{q_2}_0 \frac{\mathrm{d} q_2}{q_2-\varepsilon C_x}=-\int \limits^t_0 \frac{\mathrm{d} t}{R_2 C_x}
\end{equation}

\begin{equation}
q_x(t) = C_x \varepsilon \cdot 
\left( 
	1-
	\exp\left[		
		-\frac{t}{R_2 C_x}
	\right]
\right)
\end{equation}
\begin{equation}
U_x(t) =  \varepsilon \cdot 
\left( 
	1- 
	\exp\left[
	{-\frac{t}{R_2 C_x}}
	\right]
\right)
\end{equation}
Напряжение $U_G$ на измерительном приборе можно получить из соотношений

$\phi_1-\phi_2=U_1$; $-(\phi_2-\phi_3)=U_x$
Получаем, что
\begin{equation}
 \phi_1-\phi_3 =U_G(t)=U_1(t)-U_x(t)= \varepsilon \cdot 
 % e ^ {-\frac{t}{R_2 C_x}} - e ^{-\frac{t}{R_1 C_1}} 
	\left( 
		\exp\left[		
			-\frac{t}{R_2 C_x}
		\right]
		-
		\exp\left[		
			-\frac{t}{R_1 C_1}
		\right]
 	\right)
\end{equation}

\end{document}






















\input{\source/lab_title-page.tex} %Титульная страница

\tableofcontents
\newpage

\section*{Введение} % (fold)
\addcontentsline{toc}{section}{Введение}
\label{sec:input}

Целью данной работы является успешная сдача зачета по общефизу

\newpage
\section{Исследование неоновой лампы}
\subsection{Снятие ВАХ неоновой лампы}

\begin{table}[H]
	    \caption{Снятие вольт-амперной характеристики (ВАХ) неоновой лампы}
	    \label{tab:diod}
	    \pgfkeys{/pgf/number format/.cd,
		fixed,  1000 sep={\,}}
\newlength\Colsep
\setlength\Colsep{10pt}
\pgfplotstableset{
	multicolumn names, % allows to have multicolumn names
	col sep=tab, % the seperator in our .csv file
	precision=3,
	% fixed zerofill, 
	columns/u/.style={
	column name={$U$, В},
	},
	columns/i/.style={
	column name={$I$, мА},
	},
	empty cells with={\textbf{--}},
	every head row/.style={
	before row={\toprule},
	after row={
		\midrule}
		},
	every last row/.style={after row=\bottomrule},
	every row/.style={after row=\midrule}, 
	columns={u,i},		
	dec sep align,
	% dec zerofill
	% fixed,fixed zerofill,
	% precision=2
	}
\noindent\begin{minipage}[t][][t]{\textwidth}
\begin{minipage}[t][][b]{0.2\textwidth}
	    \pgfplotstabletypeset[
		% skip rows between index={0}{50},
		skip rows between index={25}{500},
	    ]{data/vax.csv}
\end{minipage}%\hfill
\begin{minipage}[t][][b]{0.2\textwidth}
	    \pgfplotstabletypeset[
			skip rows between index={0}{25},
			skip rows between index={50}{500},
	    ]{data/vax.csv}
\end{minipage}%
\begin{minipage}[t][][b]{0.2\textwidth}
	    \pgfplotstabletypeset[
			skip rows between index={0}{50},
			skip rows between index={75}{500},
	    ]{data/vax.csv}
\end{minipage}%
\begin{minipage}[t][][b]{0.2\textwidth}
	    \pgfplotstabletypeset[
			skip rows between index={0}{75},
			skip rows between index={100}{500},
	    ]{data/vax.csv}
\end{minipage}%
\begin{minipage}[t][][b]{0.2\textwidth}
	    \pgfplotstabletypeset[
			skip rows between index={0}{100},
			skip rows between index={125}{500},
	    ]{data/vax.csv}
\end{minipage}%
\end{minipage}		

\end{table}

\begin{figure}[tb]
	\centering
	\begin{tikzpicture}[scale=0.95]
		    \begin{axis}[
			grid=both,
			scale=2.35,
			% xmode=log,
			grid style={line width=.1pt, draw=gray!10},
			major grid style={line width=.2pt,draw=gray!50},
			% axis lines=middle,
			minor tick num=5,
			% xmin=1000, 
			% xmax=11,
			% ymin=0,
			% ymax=5,
			xlabel={$U$, В},
			ylabel={$I$, мА},
			tick style={very thick},
		    % scale=0.5,
		    grid=both,
		    grid style={line width=1pt, draw=black!20},
		    major grid style={line width=.5pt,draw=black!90},
		    minor y tick num=4,
		    minor x tick num=4,   
		    xtick distance=10,
		    ytick distance=1,
		    % ymax = 1.8,
		    xmin = 100,
		    ymin = 0,
		    % xmax = 40,
		    % ticklabel style={font=\tiny,fill=white},    
		    axis lines=middle, 	
		    xlabel style={below right, xshift=1em},
		    ylabel style={above left},		    		
			% legend style={
			% at={(rel axis cs:0,1)},
			% anchor=north west,draw=none,inner sep=0pt,fill=gray!10}
		]
	        \addplot +[mark=*, mark size=0.1em, ] table [x=u, y=i] {data/vax.csv};
	        % \addlegendentry{$U_c=-7$ В}


	        % \addplot [samples=100, domain=0:52] {0.05215*e^(-(x-27)^2/2/(7.5)^2)};
	        % \addlegendentry{$\sigma_k=7.5$}

	        % \addplot [samples=100, domain=0:52, dashdotdotted] {0.05215*e^(-(x-27)^2/2/(7.649)^2)};
	        % \addlegendentry{$\sigma_k=7.649$}
		    \end{axis}
		\end{tikzpicture}	
	\caption{Caption here}
	\label{fig:figure1}
\end{figure}


